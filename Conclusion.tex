\section{Conclusion}
We proposed a new authentication scheme called the evolving passwords based on physical controls for WLAN authentication. WLAN passwords automatically evolve at the predetermined time. A random number called physical parameter is used to update passwords. The physical parameter can only be obtained in a specific location protected by physical controls. Once a password evolves, users must pass the physical access control, enter the specific location, and get the physical parameter. After users calculating out new passwords, they can access WLAN again. A guest finishing his/her visit cannot pass the physical access control. So he/she cannot get the new password. As a result he/she cannot access WLAN again. 

A WLAN system deployed with the proposed authentication scheme consists of three parts: a master AP which can updates its own password independently, several slave APs which should interact with the master AP to update their own passwords, and many mobile devices which can automatically update their own passwords and access WLAN. At the predetermined update time, the master AP will generate a random number as a new physical parameter and calculate a new password using the old password and the new physical parameter. Then the master AP will broadcast the new physical parameter to the specific location protected by physical access controls and use the new password to authenticate mobile devices. At the same time, the slave APs will require new passwords from the master AP and then use the new password to authenticate mobile devices. As for mobile devices, they must share the same password with APs. So if APs update their passwords, they mobile devices should also update their passwords by the same way as APs. To update their own passwords, mobile devices should enter the specific location and capture the physical parameter. If they cannot update their passwords, they would inform users. 

The proposed authentication scheme can achieve fine-grained access control for users: users get the authority of WLAN access when they arrive and ask password from other users or administrators. However, their authorization will be revoked when they leave. What’s more, the proposed authentication scheme can enhance WLAN security at the same time. Meanwhile, the proposed authentication scheme has a low impact on users’ experience. WLAN passwords automatically update without participating of users and administrators. All APs in the WLAN system will update their passwords in a very short time. Password evolving have little influence on mobile devices accessing WLAN. The time it takes to update password is negligible enough compared with the time it takes to accessing WLAN. 
