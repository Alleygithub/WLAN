\section{Introduction}
Nowadays, WLAN has gained much more popularity than ever before with the popularity of mobile devices and mobile Internet because of its cheapness and fastness. We can now receive wireless signal almost everywhere such as in a park, on a bus, at home, or in workplace. WLAN becomes more and more important for us as we can surf the Internet through WLAN. On Internet, we can do a lot of things. As online office becomes increasingly popular, our work is also unable to carry out normally without network. When we visit a new place, requiring for WLAN access becomes one of the first things we do. So it is necessary for organizations where new visitors often come to deploy a special WLAN for visitors. There are two kinds of users joining such WLAN with different demands. The staff want a long-term WLAN access authorization while visitors only need to temporarily join WLAN. Also, different visitors require different time periods of WLAN access. So when involving visitors, WLAN authentication becomes a problem. It should achieve differential WLAN access control for both visitors and staff, and dynamic WLAN access control for different visitors. 

However, none of the existing solutions can realize the demand of guest WLAN - providing fine-grained access control and high security while not putting extra burden on users and administrators. For example, all users (authorized or unauthorized) can join an open guest WLAN. All messages transmitted on an open guest WLAN are in plaintext, and thus users’ private information may be revealed. Meanwhile, it is impossible to realize user access control for an open WLAN. As for password based authentication, though it can prevent unauthorized users joining WLAN, it is hard to revoke visitors’ authorization of WLAN access once they are told the password. This kind of authentication is not secure enough as passwords are guessable. The 802.1X authentication server based authentication can realize a highly secure WLAN. It can also provide very fine-grained access control for each user. However, it introduces a heavy burden on administrators as all users rely on administrators to access WLAN. Authenticating by a web portal server equipped by WIFIDog can indeed realize various demand of WLAN access control. However, this kind of authentication cannot provide any extra confidentiality and integrity protection for transmitted data. Data security still relies on the base authentication mechanism, but it is common that this kind of authentication is deployed on an open WLAN which means there is no base authentication mechanism. 

In this paper, we want to combine fine-grained access control, convenience and security for WLAN. Our goal is to achieve differential WLAN access control for visitors and staff - staff can always join WLAN while visitors can only join WLAN during their visit, and dynamic WLAN access authorization for visitors - granting when they come and revoking when they leave. Meanwhile, we do not want to put extra burden on administrators. To achieve this goal, we proposed a location-based evolving passwords scheme for WLAN authentication. WLAN passwords will automatically evolve at regular intervals. Administrators can adjust the update interval whenever needed. If passwords evolve, only authorized users can get new passwords and continue to join WLAN. This means, unauthorized users knowing old passwords will be filtered out and cannot connect to WLAN any longer. To prevent unauthorized users renewing passwords, we introduced physical access control into passwords evolving process. A random number called physical parameter is used to renewing passwords. A specific device generates physical parameters and broadcasts them to a specific location protected by a physical access control system. Thus, users can only obtain physical parameters in constrained locations. In order to obtain physical parameters and renew passwords, authorized users must pass physical access control systems. Visitors who have finished their visit will not be able to pass them. Thus, they cannot get new passwords and join WLAN, even though they may still receive wireless signal. But just as significantly, the new dynamic password scheme will not put extra burdens on authorized users and administrators. Authorized users can still get passwords from administrators or authorized users and join WLAN as usual. Passwords will evolve automatically without participation of administrators and authorized users. 

We have implemented the proposed scheme. An authorized mobile device can successfully extract physical parameters, calculate new passwords, and authenticate to APs before and after passwords evolving. There is almost no connection delay compared with the static password scheme. What authorized users and administrators need to do is just as usual. 

\textbf{Contributions.} We proposed a location-based evolving passwords scheme for WLAN authentication, providing fine-grained access control for different users while not putting extra burden on administrators and authorized users. We combine physical authentication and WLAN authentication: whether users can connect to WLAN depends on whether they can pass physical access control systems. By this way, we achieved dynamic authorization for visitors: when they come, they need to request for WLAN access authorization; when they leave, their authorization will be revoked in time. We achieved differential access control for both visitors and staff: regular staff can always pass physical access controls and update passwords. As for visitors, once they ended their visit, they would not pass physical access controls and so they would not get new passwords even though they can still receive wireless signal. 

What’s more, we enhanced the security of the password-based WLAN authentication. WLAN passwords are automatically changed from time to time. The password update interval is short enough to avoid brute force attacks. Even if an attacker gets a password, he/she will be filtered out when passwords evolve minimizing negative influences of a successful attack. 
