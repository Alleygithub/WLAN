\section{The Proposed System}
\subsection{System Overview}
We introduced a location-based evolving passwords scheme into the WPA-PSK protocol to provide fine-grained access control without extra burdens for WLANs. Mobile devices must share the same password with APs at the same time, or they will not pass the authentication of APs. However, APs’ passwords evolve once in a while. So mobile devices should always synchronize their own passwords with APs’. A long random number called physical number is used to update the password. The physical parameter can only be obtained in a specific location protected by physical access controls for mobile devices. Once APs update their passwords, users must pass physical access controls and enter the specific location to update their own passwords so that they can continue sharing the same passwords with APs. If not, they will be rejected to join the WLAN. By password evolving, visitors who have finished their visit will lose the authorization of WLAN access automatically without operation of administrators because they will no longer pass physical access controls. Besides, WLAN security improves because it becomes impossible for attackers to exhaust all passwords. 

The evolving passwords scheme with location-based physical access controls can be deployed in a big building with many visitors coming and leaving everyday. A WLAN deployed with the proposed evolving passwords scheme consists of three parts: a master AP which can update its own password independently, several slave APs which can update their own passwords by interaction with the master AP, and many mobile devices which can automatically synchronize their own passwords with APs’ and connect to APs. The master AP generates the physical parameter and broadcasts it into a specific location protected by physical access control at regular intervals. Then the master AP will calculate a new password using the old password and the new generated physical parameter and transmit the new password to slave APs securely. All APs will use the new password to authenticate mobile devices. Depending on different cases, mobile devices will use their own current passwords to join WLAN, update their own passwords and use new passwords to join WLAN, or inform users they cannot join WLAN now. Graph 1 shows the framework of the WLAN system deployed with the proposed evolving passwords scheme is displayed. 
\begin{figure}
	\begin{center}
		\includegraphics[width=\textwidth]{Overview.pdf}
		\caption{Overview of the Proposed System}
		\label{Fig:4.2}
	\end{center}
\end{figure}

For the master AP, the administrator needs to set an initial password for the master AP. Then the master AP will automatically update its passwords at regular intervals. To update password, the master AP needs to complete the following works: 
\begin{itemize}
	\item Generate a long random as a new physical parameter; 
	\item Calculate a new password using the old password and the new physical parameter; 
	\item Broadcast the new physical parameter to the specific location protected by physical access controls; 
	\item Authenticate mobile devices using the new password; 
	\item Transmit the new password to slave APs securely. 
\end{itemize}
By this way, the master AP periodically update its own password. 

As for slave APs, it is not necessary for administrators to set initial passwords for slave APs. Slave APs get the newest password from the master AP securely. Then slave APs will automatically update their own passwords in sync with the master AP. To update password, slave APs need to complete the following works: 
\begin{itemize}
	\item Require a new password from the master securely; 
	\item Authenticate mobile devices with the new password. 
\end{itemize}
By this way, slave APs always share the same password with the master AP. 

As for mobile devices, users need to get the newest password from administrators or other users through out-of-band way and input the newest password into their mobile devices. Then mobile devices can join WLAN before passwords evolving. However, if APs update their passwords, mobile devices should update their own passwords, too. To update password, mobile devices need to complete the following works: 
\begin{itemize}
	\item Capture WLAN signal and parse out the physical parameter from it; 
	\item Calculate out a new password by the same way as the master AP; 
	\item Try to join WLAN with the new password; 
	\item If joining successfully, accept the new password; 
	\item If not, reject the new password and roll back. 
\end{itemize}
After that, mobile devices can continuously join WLAN before next passwords evolving. 

\subsection{System in Detail}
The password evolving procedure can be divided into five: initial password setup and distribution, password update interval setup, physical parameter generation and broadcast, password update of APs and mobile devices. The former twos will be executed once in the initialization phase when the procedure starts, while the latter threes will be repeatedly executed in the password evolving phase at the preset time. 

\subsection{Initial Password Setup and Distribution}
For the master AP, the administrator needs to set an initial password and physical parameter for the master AP before its authenticator starts. Such information is stored in one or more configuration files. When the authenticator is launched, it reads the above information from the configuration file at first. Considering the initial password may have been expired when launched, the authenticator may need to update the initial password to current first. For the slave AP, it is unnecessary for the administrator to set initial password for its authenticator. Instead, the administrator needs to specify the IP address of the master AP. So the authenticator could acquire the current password from the master AP. When the authenticator is launched, it will require the current password from the master IP. The current password will transmitted securely from the master AP to the slave AP. As for the mobile device, the user of the mobile device needs to get the current password from the administrator or other authorized users through an out-of-band channel and store the current password in one or more configuration files of the supplicant of the mobile devices. When the supplicant is launched, it reads its own password from the configuration file. Before password evolving, the mobile device could join the WLAN as long as it can receive desired WLAN signal. 

Mobile devices can join the WLAN by the WPA-PSK protocol if they share the same password with APs. In the WPA-PSK protocol, there are two password related concepts: passphrase and pre-shared key. The passphrase is a short string usually containing 8-20 characters. It is easy to remember and type. While the pre-shared key is a 32 byte number. When people type in a passphrase, the passphrase is used to derive a pre-shared key using a pseudo random generation algorithm. Then the pre-shared key will be used in a 4-way handshake for authentication between mobile devices and APs. Like the pre-shared key, the password is also a long random number with a length of 32 bytes or even longer as it is the result of a secure hash algorithm such as SM3, SHA256. The first 32 byte of the password is used as the pre-shared key. Though it is allowed to input a pre-shared key straightforwardly, the pre-shared key is too long to type in. To make the password distribution more user-friendly, we suggest to use the QR code to distribute the pre-shared key. The pre-shared key can be displayed on the screen or printed on the paper in the form of QR code. Authorized users get the pre-shared key by scanning the QR code. As most of mobile devices are equipped with one or more camera, it will not be a big problem for authorized users. 

\subsection{Password Update Interval Setup}
Like the initial password of the master AP, the administrator should also set the password update interval before the authenticator of the master AP starts. The password update interval is also stored in the configuration file. Once launched, the authenticator will read the update interval from the configuration file together with the initial password. Meanwhile, all passwords have their expiry time. The expiry time of the initial password is set by the administrator and stored together with the initial password in the configuration file. When the authenticator reads the initial password, its expiry time will be read together. As mentioned above, the initial password in the configuration file may have been expired when the authenticator launched. The authenticator gets the current time and determines whether the initial password is expired by comparing the current time and the expiry time of the initial password. If the current time exceed the expiry time of the initial password, it should update the initial password to the current time. As for slave APs, it is unnecessary for the administrator to set the password update interval for them. When the slave AP requires the current password, its expiry time will be responded together. So the slave AP will know when the current password will be expired and when to request for a newer password. 

It is possible for the WLAN system to adopt a variable update interval. In general, the administrator can set a slightly longer update interval. However, when something big happens, the administrator can shorten the update interval. If the administrator wants to change the update interval, he/she could modify the update interval and restart the authenticator of the master AP. When the authenticator restarts, it will continue using the current password until it is expired as the current password with its expiry time is stored in the configuration file. However, when the current password is expired, the master AP will apply the new update interval. For slave APs, no matter when they require for current passwords, they will get the same password and the same expiry time. That is, modifying the update interval will not influence the update procedure of all APs in the WLAN. So it will also not influence the update procedure of mobile devices, too. 

\subsection{Physical Parameter Generation and Broadcast}
The physical parameter is a long random number which can only be acquired in a specific location protected by physical access controls for mobile devices. The physical parameter is regularly generated and broadcast by the master AP. When launched, the master AP broadcast the initial physical parameter read from the configuration file as mentioned above. During password evolving phase, the master AP will generate a long random number as a new physical parameter according to the preset password update interval and then broadcast the new physical parameter. The beacon frame is used to broadcast the physical parameter. Usually, an AP will broadcast a beacon frame at short intervals to announce here is a WLAN. There is a vendor specific field in the beacon frame which can carry custom data. So the physical parameter can be carried on the beacon frame in the vendor specific field. 

The physical parameter is broadcast through beacon frame of the master AP. Meanwhile, it should only be obtained in a constrained location. So it should be confirmed that the wireless signal of the master AP must be limited in the constrained location. Considering a wide WLAN, there shall be several APs with their signal covering all the building. Users may even receive wireless signal outside the building. However, no one outside the specific location can receive the beacon frame of the master AP and get the physical parameter. Only users who can pass the physical access control can get into the specific location, receive wireless signal of master AP, and get physical parameter from the beacon frame. 

The master AP can broadcast one or more physical parameters to adjust the frequency of authorized users entering the specific location. If the master AP broadcast one physical parameter e.g. current physical parameter, authorized users must enter the specific location in every update period. If an authorized user has not entered the specific location in one update period, he/she would never calculate out the subsequent passwords. Even if he/she enters the specific location in the next update period, he/she cannot calculate out the current password as he/she does not know the previous password which is necessary for calculating the current password. However, if the master AP broadcasts two physical parameters e.g. current and next physical parameter, it does not matter if an authorized user does not enter the specific location in the next update period after he/she enters the specific location in the current update period. Using the two physical parameters, he/she are able to calculate out the current and next password. When he/she enters the specific location in the next of the next update period, he/she could still calculate out the password. If the master AP broadcasts more physical parameter, the time span authorized users entering the specific location can be even longer. 

This feature can be applied to such a situation. Usually, the staff is required to be on duty every weekday. So the update interval can be set to one day. However, the staff may not go to work at the weekend. If the master AP still broadcasts one physical parameter on Friday, the staff cannot access WLAN on Monday. For the staff successfully accessing WLAN on Monday, the master AP should generates and broadcasts three physical parameters on Friday. 

\subsection{Password Update of APs}
All APs periodically update their own passwords. All APs use a same password to authenticate mobile devices at the same time. However, the password will not be used for a long time. At a predetermined password update time, APs will update the currently used password and then wait for the next update. 

For the master AP, once it generates a new physical parameter, it calculates the new password with the old password and the new physical parameter. Let P[i - 1] as the old password, O[i] as the new physical parameter, the new password P[i] can be calculated out as follows: $$P[i] = Hash(P[i - 1] XOR O[i])$$ The expiry time should be synchronously updated. The expiry time of the new password is the sum of the expiry time of the old password and the password update interval. Once getting the new password, the master AP should store the physical parameter and new password with its expiry time into the configuration file in case of an unexpected or planned restart. Then, it could broadcast the new physical parameter, and use the new password to authenticate mobile devices. Besides, it should response the new password with its expiry time if the slave APs request through a secure channel. 

For the slave APs, as the slave APs know the expiry time of the current password, it can request a new password through a secure channel immediately when the current password is expired. Then it could use the new password to authenticate mobile devices. As mentioned above, the password should be transmitted from the master AP to slave APs though a secure channel. Or unauthorized users may get the password though the transmission channel. For example, the administrator can establish dedicated links for transmitting the physical parameter between the master AP and the slave APs. For another example, the above information can be transmitted through the local area network(LAN). The LAN is insecure as it is connected with the WLAN and the Internet. However, the master AP can secure the channel by encrypting and adding a message authentication code before transmitted, Ensuring that only the master AP and slave APs share the secret key. It is also a proper way to establish a mutual authenticated TLS tunnel between the master AP and slave APs on the LAN and transmit the password in the TLS tunnel. 

The system time of the slave AP may be slower than that of the master AP. If so, there will be an out of sync when updating password between the master AP and the slave AP. That is, the master AP has updated the password, but the slave AP still uses the old password as it thinks the old password has not been expired yet. To achieve a near-zero time delay between the master AP and the slave AP updating passwords, All APs should regularly adjust the system time from the same time server. Besides, the slave AP could request a new password slightly earlier than the expiry time of the old password. If there is a long time delay between an slave AP and the master AP, mobile devices may not be able to connect to the slave AP as they have once received a beacon frame from the master AP and already updated the password. 

\subsection{Password Update of Mobile Devices}
If all APs update their own passwords, only mobile devices update their own passwords synchronously could they join WLAN again. Considering the update interval can be adjusted by administrators on demand, it becomes a problem how to inform mobile devices whether the password is updated and how many times the password has been updated from mobile devices getting the initial password or updating the password last time. To solve this problem, we assign a serial number for the password. The serial number starts with zero and increases by one when password evolving. APs broadcast the serial number of the password they currently use in the WPA-PSK protocol to inform mobile devices whether it needs and is able to update its own password. The serial number can be broadcast through the beacon frame using the vendor specific field like the physical parameter. For mobile devices, if the serial number of APs’ password is the same as that of their own password, it means they share the same with APs and they can join WLAN using their own password. If the serial number of AP’s password is greater by one than that of its own password, mobile devices are able to update the password if they can get the current physical parameter. If else, mobile devices cannot update the password to the current time. 

Mobile devices will regularly capture and parse beacon frames. Once capturing a frame with the specific SSID, the mobile device will parses out the serial number the authenticator currently used, and determines whether it shares the same password with the authenticator by comparing the two serial numbers. If the two serial number are the same, the supplicant can join WLAN using its own password. If the authenticator’s serial number is greater by one than its own serial number, the supplicant is able to update its own password to the same as the authenticator. If it successfully parses out a physical parameter from the beacon frame, it calculates a temporary new password without updating its own password because the physical parameter may be wrong. Then it tries to join WLAN using the temporary new password. If it successfully connect to the authenticator, it means the temporary new password is right. The supplicant should update its own password and store the temporary new password and new serial number into the configuration file in case of a planned or unexpected restart. If the supplicant fails to connect to the authenticator, it should clear the temporary new password and wait for another physical parameter. However, if the supplicant cannot parse out a physical parameter, it should inform the user passing the physical access control and entering the specific location if he/she want to join WLAN. In other cases, the application should inform the user that he/she can only get the current password from the administrator or other authorized users. 

The system time of APs and mobile devices may also be different. However, it is not a big deal. Mobile devices get the serial number of APs’ current password. If the serial number of APs’ current password does not match that of its own current password, it suggests that the password has been updated, and mobile devices must update the password if they still want to access WLAN. The system time of mobile devices does not participate in the password update process. It is better to adopt the expiry time as the serial number. Mobile devices may reject to connect to an AP if the expiry time of the AP’s current password is far less than its system time, as it suggests that the target AP may be a fishing AP controlled by an attacker who knows an old password by accidence. 
